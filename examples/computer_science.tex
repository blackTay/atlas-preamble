\documentclass[11pt, a4paper]{article}
\usepackage[english]{babel}

\usepackage[compsci]{atlas-preamble}



\title{Computer Science and \texttt{atlas-preamble}}
\author{\texttt{atlas-preamble} contributors}
\date{\today}


\begin{document}

\maketitle


\section{Introduction}
To include computer science notation macros and packages in your document, use \texttt{atlas-preamble} either with the \texttt{compsci} preset, which includes some basic and other tweaks, or use the \texttt{compsci} option on its own.

\noindent
You then have the following benefits:



\section{Theoretical notation macros}
The following often-used symbols in theoretical computer science get abbreviated macros:
\begin{align*}
	&\texttt{\textbackslash Oh\{\}} & &\texttt{\textbackslash oh\{\}} & &\texttt{\textbackslash Th\{\}} & &\texttt{\textbackslash Om\{\}} & &\texttt{\textbackslash om\{\}} \\
	&\Oh{\frac{n^2}{m}} & &\oh{\frac{n^2}{m}} & &\Th{\frac{n^2}{m}} & &\Om{\frac{n^2}{m}} & &\om{\frac{n^2}{m}}
\end{align*}



\section{Pseudocode}
For the example algorithm using \texttt{algpseudocode} from the \texttt{algorithmicx} documentation, see Algorithm \ref{alg:ex1}.

\begin{algorithm}
\label{alg:ex1}
\caption{The Bellman-Kalaba algorithm}
    \begin{algorithmic}[1]
    \Procedure{BellmanKalaba}{$G$, $u$, $l$, $p$}
        \ForAll{$v \in V(G)$}
            \State $l(v) \leftarrow \infty$
        \EndFor
        \State $l(u) \leftarrow 0$
        \Repeat
            \For{$i \leftarrow 1, n$}
                \State $min \leftarrow l(v_i)$
                \For{$j \leftarrow 1, n$}
                    \If{$min > e(v_i, v_j) + l(v_j)$}
                        \State $min \leftarrow e(v_i, v_j) + l(v_j)$
                        \State $p(i) \leftarrow v_j$
                    \EndIf
                \EndFor
                \State $l'(i) \leftarrow min$
            \EndFor
            \State $changed \leftarrow l \not= l'$
            \State $l \leftarrow l'$
        \Until{$\neg changed$}
    \EndProcedure
    
    \Statex
    \Procedure{FindPathBK}{$v$, $u$, $p$}
        \If{$v = u$}
            \State \textbf{Write} $v$
        \Else
            \State $w \leftarrow v$
            \While{$w \not= u$}
                \State \textbf{Write} $w$
                \State $w \leftarrow p(w)$
            \EndWhile
        \EndIf
    \EndProcedure
    \end{algorithmic}
\end{algorithm}

\end{document}
